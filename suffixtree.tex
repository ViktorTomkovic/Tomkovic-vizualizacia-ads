\chapter{Sufixový strom}\label{chap:sx}

\emph{Sufixový strom} je písmenkový strom pre všetky \emph{sufixy (prípony)} 
daného slova. Jeho veľká výhoda spočíva v rýchlom zostrojení a veľkom množstve 
efektívnych algoritmov na ňom. 

\section{Zostrojenie sufixového stromu}
Najjednoduchšie riešenie ako zostrojiť sufixový strom
je využiť písmenkový strom a pridať do neho všetky sufixy. Takéto riešenie má 
časovú aj pamäťovú zložitosť $O(n^2)$. Ako prvý navrhol lineárne riešenie 
\citet{weiner}. Ďalšie riešenie zostrojil \citet{McCreight}. \citet{ukkonen} 
navrhol, ako zostrojiť strom za behu. 

\section{Ukkonenov algoritmus}

Zostrojenie stromu začneme s písmenkovým stromom popísaným v predchádzajúcej 
kapitole (obrázok~\ref{img:sxk1}). Neskôr bude potrebné použiť modifikáciu 
písmenkového stromu -- strom s komprimovanými hranami. V strome s 
komprimovanými hranami môže hrana obsahovať viac ako jeden znak 
(obrázok~\ref{img:sxk3}).

Často budeme chcieť označovať podslovo slova $w = w_1w_2w_3\cdots w_n$. 
Pre podslovo od indexu $i$ po index $j$ sme zaviedli označenie $w[i:j]$, teda 
$w = w[1:n]$. Keďže každá cesta z koreňa do nejakého vrcholu zodpovedá slovu, 
vrcholy stotožníme so slovami. Pokiaľ hovoríme, že sme slovu $w$ 
pripojili písmenko $a$, myslíme tým, že sme do písmenkového stromu za 
korešpondujúci vrchol pripojili ďalší s hranou $a$. V prípade stromu s 
komprimovanými hranami to znamená, že sme hranu rozšírili o písmenko $a$.

Hlavnou myšlienkou algoritmu je pre slovo $w = w_1w_2w_3\cdots w_n\uz$ 
postupne vytvoriť sufixové stromy pre slová $w[1:1], w[1:2], w[1:3], \ldots, 
w[1:n+1]$. Označme tieto stromy $T(1), T(2),\ldots, T(n), T(n+1) = T$. 
Teda, vytvárame $n+1$ stromov a pomocou písmenkového stromu s nekomprimovanými 
hranami, každý zatiaľ na vytvorenie potrebuje $O(n^2)$ 
krokov. Toto je na prvý pohľad veľmi nešikovné riešenie, pretože počet 
krokov potrebných na zostrojenie všetkých stromov je $O(n^3)$. 
Avšak zopár vylepšeniami je možné tento algoritmus zrýchliť až na hranicu 
$O(n)$ a to aj pre rýchlosť, aj pre pamäť. 
% well, nikde nie je napísané, že je to theta, ale hádam potrebujeme aspoň 
% prečítať vstup, nie?

Môžeme si všimnúť, že keď už máme vytvorený strom 
$T(i-1)$ (ktorý obsahuje sufixy $w[1:{i-1}], w[2:{i-1}], \ldots, 
w[{i-1}:{i-1}]$), tak pre vytvorenie stromu $T(i)$ netreba znovu vytvárať nový 
strom a pridávať do neho sufixy $w[1:i], w[2:i], \ldots, w[{i-1}:i], w[i:i]$, 
ale stačí rozšíriť existujúce sufixy stromu $T(n-1)$ o znak $w_i$.

Druhým dobrým pozorovaním je fakt, že keď máme znak $a$, slovo $v$ 
a v strome sa nachádza sufix $av$, tak sa v strome určite 
nachádza aj slovo $v$. Vyplýva to zo základnej vlastnosti sufixových stromov 
--  sufixový strom obsahuje všetky sufixy.

\def\wOne{0.6}
\begin{figure}
\centering
\subfloat[Písmenkový strom.]{
\label{img:sxk1}
\includegraphics[scale=\wOne]{obrazky/sxkomp3.png}
}
\quad
\subfloat[Sufixový strom s nekomprimovanými hranami.]{
\label{img:sxk2}
\includegraphics[scale=\wOne]{obrazky/sxkomp2.png}
}
\quad
\subfloat[Sufixový strom.]{
\label{img:sxk3}
\includegraphics[scale=\wOne]{obrazky/sxkomp1.png}
}
\caption{\emph{Rôzne typy sufixových stromov.}}
\label{img:sxk}
\end{figure}
 
\subsection{Sufixové linky}

Pri vytváraní stromu $T(i)$ zo stromu $T(i-1)$ postupne rozširujeme sufixy, no 
bolo by neefektívne ich vždy vyhľadávať z koreňa. Preto zavedieme 
\emph{sufixovú linku}, čo je orientovaná hrana, ktorú má každý vrchol okrem 
koreňa. Zo sufixu $av$ ide linka do sufixu $v$. 

Algoritmus teda zmeníme tak, že namiesto opätovného vyhľadávania sufixu z 
koreňa, vyhľadáme bod, z ktorého začneme pridávať len raz a ďalej sa 
navigujeme po linkách. 

Pri vytváraní stromu $T(i)$ ako prvý pridávame najdlhší sufix $w[1:i]$. Sufix 
$w[1:i-1]$ v~strome máme a pamätáme si ho ako miesto, z~ktorého vytváranie 
stromu $T(i)$ začneme. Rozšírime sufix $w[1:i-1]$ o~písmenko $w_i$. Ďalej by sme 
chceli rozšíriť sufix $w[2:i-1]$. Ten sa v~strome nachádza a smeruje naňho 
sufixová linka. Stačí po nej prejsť, rozšíriť sufix $w[2:i-1]$ o písmenko 
$w_i$ a vytvoriť sufixovú linku z $w[1:i]$ do $w[2:i]$. Takto pokračujeme, až 
kým nepripojíme sufix $w[i:i]$.

Ako nám to pomohlo zlepšiť časovú zložitosť? Miesto, z ktorého štartujeme 
pridávanie sufixov si pamätáme. Rozšírenie stromu $T(i-1)$ o jeden znak 
zaberie $O(i)$ krokov. Postupne rozširujeme $n$ ráz. Na vybudovanie stromu $T$ 
teda potrebujeme $O(n^2)$ krokov.

\subsection{Zníženie nárokov na pamäť}

Veľmi priamočiarym krokom k zlepšeniu pamäťovej náročnosti sa zdá byť 
kompresia hrán. Takto má celý strom dovedna 
$O(n)$ hrán. Namiesto toho, aby sme si na hrane 
pamätali celý reťazec, budeme si na nej pamätať len počiatočnú a konečnú 
pozíciu v slove, pre ktorý beží algoritmus\footnote{Tento krok predpokladá, 
že vieme adresovať reťazec priamo.}. Po týchto zlepšeniach nám stačí 
pamätať si $O(n)$ informácií. 

Keďže kompresiou sa niektoré vrcholy stratili, musíme upraviť spôsob, ako sa 
pohybovať po sufixových linkách. Úprava bude mierna. Po rozšírení sufixu 
$w[i:j]$ o znak prejdeme do najbližšieho vrcholu a popri tom si zapamätáme 
reťazec znakov $\alpha$ na hrane, ktorou sme prešli. Keďže si znaky na hrane 
pamätáme ako dvojicu čísel, toto nám zaberie len $O(1)$ krokov. Keď skončíme 
v koreni musíme vyhľadať celý reťazec $w[i+1:j]$. Inak 
prejdeme po sufixovej linke, vyhľadáme zapamätaný reťazec. Z vlastností stromu 
vyplýva, že stačí pozerať len na prvé písmeno každej hrany preto, aby sme 
vedeli, či po nej máme prejsť alebo nie.

To, akým spôsobom sa písmenko po vyhľadaní pridá a kde vieme ušetriť počet 
krokov, popíšeme v nasledujúcej časti.

\subsection{Rozširovanie stromu}

Pri rozširovaní stromu o písmenko môžu nastať tieto tri prípady:
\begin{itemize}
\item \emph{prvý prípad:} písmenko pripájame k vrcholu, z ktorého nepokračuje 
hrana, a teda je listom;
\item \emph{druhý prípad:} písmenko pripájame na miesto, z ktorého nepokračuje 
hrana s daným písmenkom, ale pokračuje z neho hrana s iným písmenkom;
\item \emph{tretí prípad:} písmenko pripájame na miesto, z ktorého pokračuje 
hrana s daným písmenkom.
\end{itemize}

Po tomto rozdelení môžeme ľahšie pozorovať ďalšie veci. Prvou je, že akonáhle 
vytvoríme novú vetvu (nastane prvý alebo druhý prípad), tak ju v ďalších 
krokoch môžeme len rozšíriť. Čiže, ak sme raz pridali list, ten aj listom 
navždy ostane. Túto skutočnosť môžeme využiť v implementácií tak, že pri 
vytváraní hrany 
nastavíme indexy hrany na $i$ (index momentálneho písmenka, o ktoré strom 
rozširujeme) a $e$, čo je globálna premenná označujúca momentálny koniec v 
slove, ktorá sa každým rozšírením zvýši o jeden.

Druhým pozorovaním je, že po prvom výskyte tretieho prípadu sa strom už 
nerozšíri. Vyplýva to zo základnej vlastnosti sufixového stromu: ak 
rozširujeme strom o písmenko $w_i$ a v strome je slovo $w[j:i]$, tak 
sú v ňom aj slová $w[j+1:i], w[j+2:i], \ldots, w[i:i]$.

Ďalším pozorovaním je, že pri rozširovaní najprv nastávajú prvé prípady, 
potom druhé a až potom tretie.

So všetkými týmito poznatkami môžeme konečne skonštruovať výslednú podobu 
algoritmu.

\def\wTwo{0.65}
\begin{figure}
\centering
\subfloat[]{
\label{img:sxr1}
\includegraphics[scale=\wTwo]{obrazky/sx5.png}
}
\quad
\subfloat[]{
\label{img:sxr2}
\includegraphics[scale=\wTwo]{obrazky/sx61.png}
}
\quad
\subfloat[]{
\label{img:sxr3}
\includegraphics[scale=\wTwo]{obrazky/sx62.png}
}

%\quad
\subfloat[$T(4)$.]{
\label{img:sxr4}
\includegraphics[scale=\wTwo]{obrazky/sx63.png}
}
\qquad
\subfloat[$T(5)$.]{
\label{img:sxr5}
\includegraphics[scale=\wTwo]{obrazky/sx65.png}
}
\caption{\emph{Rozšírenie stromu o písmenko~{\tt A}.} 
\subref{img:sxr1}~Sufixový strom pre slovo {\tt ABBAB}. 
\subref{img:sxr2}~Vykonali sme rozšírenie prvých prípadov -- zvýšili sme index o 
jeden. Šli sme o hranu vyššie, zapamätali sme si, čo je na nej. Prešli sme 
po sufixovej linke a ideme hľadať zapamätaný reťazec. Keďže hľadáme z koreňa, 
zapamätaný reťazec nám nepomôže. Musíme nájsť celý reťazec {w[4:5]}. 
\subref{img:sxr3}~Našli sme reťazec, nastal druhý prípad. Rozdelili sme hranu a 
vytvorili nový vrchol. Teraz musíme ísť po hrane vyššie, vyhľadať nové miesto 
pripojenia a vytvoriť sufixovú linku. 
\subref{img:sxr4}~Našli sme miesto pripojenia a nastal tretí prípad: písmenko sa 
nachádza na niektorej z~hrán. Vytrvoríme sufixovú linku a skončíme. 
\subref{img:sxr5}~Strom po rozšírení o písmenko~{\tt A}.
}
\label{img:sxr}
\end{figure}

\subsection{Zhrnutie}

Sufixový strom $T$ pre slovo $w = w_1w_2w_3\cdots w_n\uz$ vytvoríme tak, že z 
prázdneho stromu vytvoríme postupne stromy $T(1), T(2), T(3),\ldots, T(n), 
T(n+1) = T$. 

Strom $T(1)$ vytvoríme triviálne; pridáme koreňu hranu s dvojicou indexov 
$(1, e)$ a nastavíme štartovací vrchol $u_s$ na práve vytvorený vrchol.

Postupne vytvárame strom $T(i)$ zo stromu $T(i-1)$ rozšírením stromu 
$T(i-1)$ o písmenko $w_i$ následovne:

\newpage
\begin{enumerate}
\item index $e$ nastavíme na novú hodnotu $i$; 
Tým sme vykonali všetky prvé prípady. Vieme si udržiavať ich počet $j$;
\item za aktuálny vrchol $u$ si zvolíme štartovací vrchol $u_s$;
\item z aktuálneho vrcholu prejdeme po hrane vyššie a zapamätáme si reťazec 
na hrane po ktorej sme prešli;
\item ak nie sme v koreni, prejdeme po sufixovej linke. Nastavíme aktuálny 
vrchol;
\item z aktuálneho vrcholu vyhľadáme sufix $w[j+1:i]$ (ak sme v koreni 
potrenujeme nájsť celý sufix $w[j+1:i]$, ak sme v inom vrchole stačí 
vyhľadať zapamätaný reťazec z 3. kroku);
\item teraz môže nastať druhý alebo tretí prípad:
\begin{itemize}
\item ak nastal druhý prípad, v prípade potreby rozdelíme hranu. Nastavíme 
aktuálny vrchol na posledný navštívený (respektíve vytvorený) a v prípade 
potreby nastavíme sufixovú linku. K aktívnemu vrcholu pripojíme hranu s 
indexami $(i,e)$, zvýšime index $j$ o 1 a nastavíme štartovací vrchol na 
práve vytvorený. Vrátime sa na krok~3;
\item ak nastal tretí prípad, v prípade potreby nastavíme sufixovú linku.
\end{itemize}
\item zvýšime hodnotu $i$ o jeden a vrátime sa na krok 1.
\end{enumerate}
 
%Takto vybudujeme sufixový strom pre slovo $w$.

\section{Použitie}\label{sec:sx:usage}

Sufixový strom má v stringológií veľa využití \citep{gusfield}. Algoritmus na 
zistenie prítomnosti podreťazca v slove je rovnaký ako pri písmenkovom strome. 
Navyše vieme určiť, na ktorej pozícií sa podreťazec vyskytuje. Pri 
konštrukcii stromu stačí označovať listy číslami v poradí podľa vytvorenia. 
Hľadanie podreťazca~$p$ v predspracovanom texte~$T$ nám zaberie len 
$O(|p|)$~krokov.

Sufixový strom pre viacej reťazcov sa nazýva \emph{všeobecný sufixový strom} 
a dá sa zostrojiť miernou úpravou Ukkonenovho algoritmu. Reťazce $v_1, v_2, 
v_3, \ldots, v_n$ spojíme a oddelíme unikátnymi oddelovačmi. Vznikne nám slovo 
$v_1\uz_1v_2\uz_2v_3\uz_3\cdots v_n\uz_n$, z ktorého vieme vybudovať sufixový 
strom. Tento strom sa dá využiť na vyhľadanie \emph{najdlhšieho spoločného 
podreťazca}. 

\emph{Najdlhší opakujúci sa podreťazec} sa v sufixovom strome určite vetví, 
takže ho nájdeme ako najdlhšie podslovo, ktoré sa nekončí v liste.

Pre slovo $\alpha$ vieme pomocou sufixových stromov nájsť efektívne 
\emph{najdlhšie palindromické podslovo}, čo je slovo $\beta \subset \alpha$ 
také, že $\beta^R = \beta$. Zostrojíme všeobecný sufixový strom pre slová 
$\alpha$ a $\alpha^R$. Pre tento strom vieme efektívne zostrojiť dátovú 
štruktúru pre najnižšieho spoločného predka \citep{lca}. Potom nám stačí 
pre každý index $q$ od $1$ po $n-1$ hľadať najnižšieho spoločného predka pre 
sufixy $w[q:n]$ a $w^R[n-q:n]$.

\def\wThr{0.78}
\begin{figure}
\centering
\subfloat[$T(1)$.]{
\label{img:sxsx1}
\includegraphics[scale=\wThr]{obrazky/sx1.png}
}
\quad
\subfloat[$T(2)$.]{
\label{img:sxsx2}
\includegraphics[scale=\wThr]{obrazky/sx2.png}
}
\quad
\subfloat[$T(3)$.]{
\label{img:sxsx3}
\includegraphics[scale=\wThr]{obrazky/sx3.png}
}
\quad
\subfloat[$T(4)$.]{
\label{img:sxsx4}
\includegraphics[scale=\wThr]{obrazky/sx4.png}
}

%\quad
\subfloat[$T(5)$.]{
\label{img:sxsx5}
\includegraphics[scale=\wThr]{obrazky/sx5.png}
}
\quad
\subfloat[$T(6)$.]{
\label{img:sxsx6}
\includegraphics[scale=\wThr]{obrazky/sx6.png}
}
\qquad
\subfloat[$T(7)$.]{
\label{img:sxsx7}
\includegraphics[scale=\wThr]{obrazky/sx7.png}
}

%\quad
\subfloat[$T(8)$.]{
\label{img:sxsx8}
\includegraphics[scale=\wThr]{obrazky/sx8.png}
}
\quad
\subfloat[$T = T(9)$.]{
\label{img:sxsx9}
\includegraphics[scale=\wThr]{obrazky/sx10.png}
}
\caption{\emph{Sufixové stromy.} Na obrázku sú sufixové stromy pre slová 
{\tt A}, {\tt AB}, {\tt ABB}, {\tt ABBA}, {\tt ABBAB}, 
{\tt ABBABA}, {\tt ABBABAA}, {\tt ABBABAAB}, {\tt ABBABAAB\uz}. Šípky 
znázorňujú sufixové linky, ružovou a zelenou sú označené hrany, na ktoré 
sa bude v nasledujúcom, kroku aplikovať prvé pravidlo. Čísla označujú, na 
ktorom znaku začína daný sufix v slove.}
\label{img:sxsx}
\end{figure}

% \section{Vizualizácia}
% 
% Sufixový strom sme vizualizovali podobne ako písmenkový strom, použili sme 
% Walkerov algoritmus \citep{walker} so zakrivenými hranami. Sufixové linky 
% sme znázornili šedými šípkami. Na rozdiel od písmenkového stromu sa v 
% sufixovom strome vyskytujú na hranách reťazce. Tie sme znázornili ako viac 
% spojených hrán.
% 
% Pri vizualizácií algoritmu oznažujeme hrany, pre ktoré platí prvý prípad, 
% ružovou farbou. Hranu, pre ktorú platí prvý prípad a je pridaná ako posledná, 
% označujeme zelenou farbou. Z tejto hrany začína "zaujímavejšia" časť 
% rozširovania stromu.


