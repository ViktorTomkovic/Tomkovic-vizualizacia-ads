\chapter{Implementácia softvéru}\label{chap:implementacia}

V tejto kapitole postupne uvedieme realizáciu návrhu popísaného 
v~kapitole~\ref{chap:popis}. Popíšeme ako sme vizualizovali dané 
dátové štruktúry a algoritmy (sekcia \ref{sec:im:vis}) a ukážeme si 
implementáciu a ovládanie aplikácie (sekcia \ref{sec:im:im}). 
Na záver uvedieme ako dopadlo testovanie softvéru (sekcia \ref{sec:im:test}).
% \todo{
% Programovo realizujeme podrobný návrh databázovej aplikácie. 
% Vypracuje sa podrobná dokumentácia k vytvorenému produktu. 
% Vytvorí sa používateľská príručka, kde sa podrobne dokumentuje používateľské rozhranie, spresní sa popis funkcií a spôsob ich aktivácie.
% }

\section{Vizualizácia}\label{sec:im:vis}

Základom pre vizualizáciu dátových štruktúr bol upravený algoritmus pre 
tesné vykresľovanie a použitie rôznych farieb pre prehľadnejšiu vizualizáciu.

\begin{figure}
\centering
\includegraphics[scale=0.63]{obrazky/ufshow}
\caption{\emph{Vizualizácia union-find.} Jednotlivé množiny sú vykresľované 
tesne a sú od seba oddelené pomyselnou zvislou čiarou.}
\label{img:ufshow}
\end{figure}

\subsection{Union-find}

Dátovú štruktúru union-find sme vizualizovali ako les. Pre názorné 
oddelenie množín sme si zvolili pravidlo, ktoré zakazovalo vykresliť vrchol 
z iného stromu (prvok inej množiny) 
napravo od najľavejšieho vrcholu a naľavo od napravejšieho vrcholu inej 
množiny. Jednotlivé množiny sme už vykreslovali tesným Walkerovým algoritmom 
\citep{walker}. Vykresľovanie je znázornené na obrázku~\ref{img:ufshow}.

Vizualizácia poskytuje všetky heuristiky na spájanie a 
nájdenie reprezentanta množiny a 
aj tlačidlo na vykonanie viacerých náhodných spojení naraz. Toto je užitočné, 
keď chce používateľ vidieť, ako dátová štruktúra vyzerá, po vykonaní 
veľkého počtu operácií. Okrem bežného textového vstupu je možné vyberať prvky aj myšou. 
Vybrané prvky sú zvýraznené.

\subsubsection{Operácia $\find$}

Hľadanie reprezentanta množiny sme vizualizovali tak, že sme vybratý prvok 
označili, vyznačili sme cestu do koreňa a reprezentanta množiny 
(algoritmus dopredu nepozná reprezentanta, ale v rámci vizualizácie sme to 
považovali za vhodné). Následne sme vykonali kompresiu, aktívny vrchol je 
modrý. Cestu sme nechali 
znázornenú šedou farbou, aby bolo vidieť, ako vyzerá cesta pred a po prevedení 
algoritmu. 

Pre heuristiky delenie a pólenie cesty sme použili ružovú farbu na označenie 
syna a vnuka, keďže tieto algoritmy s nimi pracujú. Operácia $\find$ je 
znázornená na obrázku~\ref{img:kompresia}.

\subsubsection{Operácia $\union$}

Spájanie prvkov sme vizualizovali vykonaním dvoch hľadaní reprezentanta a 
následného napojenia reprezentantov podľa typu algoritmu.

\begin{figure}
\centering
\includegraphics[scale=0.80]{obrazky/trie}
\caption{\emph{Vizualizácia písmenkového stromu.} Písmenkový strom obsahuje 
slová {\tt ADAM\uz, HODIL\uz, JAGAVU\uz, JAHODU\uz, JASNE\uz, K\uz, MAGDE\uz, 
MALUJUCEJ\uz, POKRAJANU\uz, POLOVICU\uz, POMARANCA\uz, S\uz, TRPKYMI\uz, 
VYSNAMI\uz}.}
\label{img:trieshow}
\end{figure}


\subsection{Písmenkový strom}

Pri vizualizácií písmenkového stromu sme použili 
Walkerov algoritmus pre úsporné rozloženie vrcholov v strome
\citep{walker}. Keď má vrchol viacej synov a hrany kreslíme priamo, tak vzniká 
nedostatok priestoru pre umiestnenie znakov na hrany. Preto sme sa rozhodli 
kresliť hrany zakrivene, podľa Bézierovej krivky. 
Vo vizualizácií sa dajú vložiť náhodné slová podľa momentálne nastaveného 
jazyka. Taktiež sa automaticky odstraňuje diakritika a interpunkcia, takže 
sa dá naraz vložiť súvislý text (obrázok~\ref{img:trieshow}).

\subsubsection{Operácia $\trieinsert$}

Aby bolo zreteľné, aké slovo vkladáme, znázorňujeme zbytok vkladaného slova 
pri mieste, kde sa práve v strome nachádzame. Časť, ktorú sme vložili a aj 
časť, ktorú ideme vložiť vypisujeme bielou na modrom pozadí. Miesto, na ktoré 
sa presunieme, prípadne kam ideme pripájať, sme označili ružovou farbou. 
Ukončovací znak (\uz) sme vykresľovali tmavšie a menším písmom.

\begin{figure}
\centering
\includegraphics[scale=0.55]{obrazky/treexfind}
\caption{\emph{Hľadanie slova {\tt MORE\uz}.} Ružovou je vyznačené hľadané 
slovo. Šedou je označená množina písmen, na ktoré sa algoritmus práve 
pozerá.}
\label{img:triefind}
\end{figure}

\subsubsection{Operácia $\find$}

Podobne ako pri vkladaní slova, aj pri vizualizácií hľadania slova, sme 
použili pomocnú bublinu so zbytkom nespracovaného vstupu. Na rozdiel od 
vkladania sme použili ružovú farbu. Na znázornenie množiny prvkov, kde sa môže 
vyskytovať následujúce písmenko sme použili šedú farbu 
(obrázok~\ref{img:triefind}).

\subsubsection{Operácia $\delete$}

Mazanie slova z písmenkového stromu sme implementovali ako nájdenie slova a 
následné zmazanie prípadnej mŕtvej vetvy (popísané v sekcii 
\ref{sec:trie:popis}). Na nájdenie slova sme použili ten istý farebný princíp, 
ako pri operácií $\find$. Mŕtvu vetvu sme zvýraznili červenou farbou 
(obrázok~\ref{img:triedelete}).

\subsection{Sufixový strom}

Sufixový strom sme vizualizovali podobne ako písmenkový strom, použili sme 
Walkerov algoritmus \citep{walker} so zakrivenými hranami. Sufixové linky 
sme znázornili šedými šípkami. Na rozdiel od písmenkového stromu sa v 
sufixovom strome vyskytujú na hranách reťazce. Tie sme znázornili ako viac 
spojených hrán a text sme spojili.

\subsubsection{Ukkonenov algoritmus}

Pri vizualizácií algoritmu (obrázok~\ref{img:sxsx}) označujeme hrany, pre 
ktoré platí prvý prípad, 
ružovou farbou. Hranu, pre ktorú platí prvý prípad a je pridaná ako posledná, 
označujeme zelenou farbou. Z tejto hrany začína "zaujímavejšia" časť 
rozširovania stromu. Pri pridávaní sufixov do stromu sme použili modrú farbu.

\begin{figure}
\centering
\greybox{\includegraphics[scale=0.4]{obrazky/guigui}}
\caption{\emph{Grafické rozhranie softvéru Gnarley Trees.} V hornej časti je 
lišta s menu pre výber dátových štruktúr a voľbu jazyka. Hlavnú časť 
obrazovky tvorí priestor na vykresľovanie dátových štruktúr. Vpravo je 
priestor na popis udalostí. Dole sú ovládacie tlačidlá.}
\label{img:gui}
\end{figure}

\newpage
\section{Popis ovládania}\label{sec:im:im}

Obrazovka softvéru sa skladá z menu na výber dátových štruktúr, menu na výber 
jazyk, obrazovky na vykresľovanie, ovládacích tlačidiel a obrazovky na výpis 
komentárov (obrázok~\ref{img:gui}).

Keď si z menu vyberieme dátovú štruktúru zobrazí sa inicializovaná dátová 
štruktúra a príslušné tlačidlá. Celá dátová štruktúra sa dá posúvať ťahaním 
myšou, približovať alebo odďaľovať použitím kolieska myši, alebo automaticky 
vystrediť a priblížiť pomocou tlačidla.

\subsection{Union-find}

\paragraph{Vstup.} Vstup zadávame do vstupného poľa alebo vyberieme prvok 
myšou. Prioritu majú prvky vybrané myšou. Pre operáciu $\find$ sa vyberie prvý 
prvok, pre operáciu $\union$ sa vyberú prvé dva prvky. Pokiaľ je zadaný menší 
počet prvkov náhodne sa doplnia. 

\paragraph{Náhodné spojenie.} Tlačidlo pre vykonanie viacerých náhodných 
spojení berie parameter -- počet spojení zo vstupného poľa. Pokiaľ nie je 
hodnota zadaná, za parameter sa berie základná hodnota.

\subsection{Písmenkový a sufixový strom}

\paragraph{Vstup.} Do vstupného poľa zadávame slová oddelené medzerou. Textový 
vstup sa pred pridávaním ošetrí: odstráni sa diakritika, interpunkcia a čísla. 
Následne sa text prevedie na veľké písmená a za každé slovo sa pridá 
ukončovací znak (\uz). Prázdne slovo sa do stromu dá pridať zadaním vstupného 
reťazca "\uz". Pokiaľ je vstup prázdny, vykoná sa pridanie náhodného slova 
podľa nastaveného jazyka.

\paragraph{Náhodné vloženie.} Tlačidlo pre vykonanie viacerých náhodných 
pridaní funguje ako pri algoritme union-find. Slová, ktoré pridá, sú vybraté 
z~databázy slov nastaveného jazyka. Sufixový strom sa vytvára pre jedno 
slovo, a preto nemá zmysel pridávať viacero náhodných slov.


\section{Testovanie, prevádzka a údržba}\label{sec:im:test}

Pri testovaní aplikácie sa nevyskytli nečakané chyby. Za najväčšie nedostatky 
sme považovali občas nezrozumiteľné komentáre a neoptimálne znázornenie 
Ukkonenovho algoritmu. Pri premietaní cez projektor sme prišli na nedostatok: 
niektoré farby sú príliš sýte, a preto sa kontrast medzi písmom a pozadím zdá 
byť malý a farby príliš tmavé. Tento problém sa vyskytoval len občas, ale 
pri bežnom osvetlení.

Softvér nepotrebuje špecifickú údržbu, keďže sa jedná o samostatnú aplikáciu. 
Avšak vhodné je sledovať aktualizácie, ktoré opravujú chyby. Vývoj a aj 
najnovšie verzie sú voľne dostupné na systéme pre správu zdrojových kódov 
Github, konkrétne na webovej stránke: \url{https://github.com/kuk0/alg-vis}. 
Stabilná verzia s popisom algoritmov sa nachádza na stránke: 
\url{http://people.ksp.sk/~kuko/gnarley-trees}. Na oboch sídlach sa nachádzajú 
kontakty, ktorými môže používateľ upozorniť na chyby v aplikácií.
% \todo{
% Testovanie sa vykonáva v prevádzkovom prostredí zvyčajne u používateľa. 
% Vykonávame testovanie jednotlivých  dekomponovaných častí a chovanie aplikácie ako celku. 
% Overujeme, či vytvorený produkt zodpovedá špecifikácii a spľňa požiadavky používateľa
% }

% \todo{
% Prevádzku má na starosti používateľ, ktorý sleduje situácie pri ktorých vznikli chyby. 
% Ďalší rozvoj systému a opravy sleduje jeho tvorca - dodávateľ.
% } 

