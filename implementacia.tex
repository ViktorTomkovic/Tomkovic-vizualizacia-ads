\chapter{Implementácia softvéru}\label{chap:implementacia}

V tejto kapitole postupne uvedieme realizáciu návrhu popísaného 
v~kapitole~\ref{chap:popis}. Najskôr si popíšeme ako sme vizualizovali dané 
dátové štruktúry a algoritmy (sekcia \ref{sec:im:vis}) a ukážeme si 
implementáciu vybraných tried (sekcia \ref{sec:im:im}). 
Na záver uvedieme ako dopadlo testovanie softvéru (sekcia \ref{sec:im:test}).
% \todo{
% Programovo realizujeme podrobný návrh databázovej aplikácie. 
% Vypracuje sa podrobná dokumentácia k vytvorenému produktu. 
% Vytvorí sa používateľská príručka, kde sa podrobne dokumentuje používateľské rozhranie, spresní sa popis funkcií a spôsob ich aktivácie.
% }

\section{Vizualizácia}\label{sec:im:vis}

Dátové štruktúry sme vizualizovali rôzne. Základom bol upravený algoritmus pre 
tesné vykresľovanie a použitie rôznych farieb pre prehľadnejšiu vizualizáciu.

\subsection{Union-find}

Dátovú štruktúru union-find sme vizualizovali ako les. Pre názorné 
oddelenie množín sme si zvolili pravidlo, ktoré zakazovalo vykresliť vrchol 
z iného stromu (prvok inej množiny) 
napravo od najľavejšieho vrcholu a naľavo od napravejšieho vrcholu inej 
množiny. Jednotlivé množiny sme už vykreslovali tesným Walkerovým algoritmom 
\citep{walker}. 

Vizualizácia poskytuje všetky heuristiky na spájanie a 
nájdenie reprezentanta množiny a 
aj tlačidlo na vykonanie viacerých náhodných spojení naraz. Toto je užitočné, 
keď chce uživateľ vidieť, ako dátová štruktúra vyzerá, po vykonaní 
veľa operácií. Okrem bežného textového vstupu je možné vyberať prvky aj myšou. 
Vybrané prvky sú zvýraznené.

\subsubsection{Operácia $\find$}

Hľadanie reprezentanta množiny sme vizualizovali tak, že sme vybratý prvok 
označili, vyznačili sme cestu do koreňa a reprezentanta množiny 
(algoritmus dopredu nepozná reprezentanta, ale v rámci vizualizácie sme to 
považovali za vhodné). Následne sme vykonali kompresiu. Cestu sme nechali 
znázornenú šedou farbou, aby bolo vidieť ako vyzerá cesta pred a po prevedení 
algoritmu. 

Pre heuristiky delenie a pólenie cesty sme použili ružovú farbu na označenie 
syna a vnuka, keďže tieto algoritmy s nimi pracujú.

\subsubsection{Operácia $\union$}

Spájanie prvkov sme vizualizovali vykonaním dvoch hľadaní reprezentanta a 
následného napojenia reprezentantov podľa typu algoritmu.

\subsection{Písmenkový strom}

Pri vizualizácií písmenkového stromu sme použili 
Walkerov algoritumus pre úsporné rozloženie vrcholov v strome
\citep{walker}. Keď má vrchol viacej synov a hrany kreslíme priamo, tak vzniká 
nedostatok priestoru pre umiestnenie znakov na hrany. Preto sme sa rozhodli 
kresliť hrany zakrivene, podľa Bézierovej krivky určenej štyrmi bodmi. 
Vo vizualizácií sa dajú vložiť náhodné slová podľa momentálne nastaveného 
jazyka. Taktiež sa automaticky odstraňuje diakritika a interpunkcia, takže 
sa dá naraz vložiť súvislý text.

\subsubsection{Operácia $\put$}

\subsubsection{Operácia $\find$}

\subsubsection{Operácia $\delete$}

\subsection{Sufixový strom}

Sufixový strom sme vizualizovali podobne ako písmenkový strom, použili sme 
Walkerov algoritmus \citep{walker} so zakrivenými hranami. Sufixové linky 
sme znázornili šedými šípkami. Na rozdiel od písmenkového stromu sa v 
sufixovom strome vyskytujú na hranách reťazce. Tie sme znázornili ako viac 
spojených hrán.

\subsubsection{Ukkonenov algoritmus}

Pri vizualizácií algoritmu oznažujeme hrany, pre ktoré platí prvý prípad, 
ružovou farbou. Hranu, pre ktorú platí prvý prípad a je pridaná ako posledná, 
označujeme zelenou farbou. Z tejto hrany začína "zaujímavejšia" časť 
rozširovania stromu.

\section{Implementácia tried}\label{sec:im:im}


\section{Testovanie, prevádzka a údržba}\label{sec:im:test}

% \todo{
% Testovanie sa vykonáva v prevádzkovom prostredí zvyčajne u používateľa. 
% Vykonávame testovanie jednotlivých  dekomponovaných častí a chovanie aplikácie ako celku. 
% Overujeme, či vytvorený produkt zodpovedá špecifikácii a spľňa požiadavky používateľa
% }

% \todo{
% Prevádzku má na starosti používateľ, ktorý sleduje situácie pri ktorých vznikli chyby. 
% Ďalší rozvoj systému a opravy sleduje jeho tvorca - dodávateľ.
% } 

